\documentclass[12pt]{article}
\newtheorem{theorem}{Theorem}
\usepackage{graphicx}
\usepackage{float}
\usepackage{url}
\usepackage{hyperref}
\usepackage{subcaption}
\usepackage[style=numeric,sorting=none,maxbibnames=9,autopunct=true,babel=hyphen,hyperref=true,backend=biber]{biblatex}

\usepackage{xcolor}
\usepackage{lmodern}
\usepackage{listings}
\definecolor{mygreen}{rgb}{0,0.6,0}
\definecolor{mygray}{rgb}{0.5,0.5,0.5}
\definecolor{mymauve}{rgb}{0.58,0,0.82}
\lstset{
  basicstyle=\footnotesize,        % the size of the fonts that are used for the code
  breakatwhitespace=false,         % sets if automatic breaks should only happen at whitespace
  breaklines=false,                 % sets automatic line breaking
  captionpos=b,                    % sets the caption-position to bottom
  commentstyle=\color{mygreen},    % comment style
  extendedchars=true,              % lets you use non-ASCII characters; for 8-bits encodings only, does not work with UTF-8
  keepspaces=true,                 % keeps spaces in text, useful for keeping indentation of code (possibly needs columns=flexible)
  keywordstyle=\color{blue},       % keyword style
  language=[90]Fortran,                 % the language of the code
  numbers=left,                    % where to put the line-numbers; possible values are (none, left, right)
  numbersep=5pt,                   % how far the line-numbers are from the code
  numberstyle=\tiny\color{mygray}, % the style that is used for the line-numbers
  rulecolor=\color{black},         % if not set, the frame-color may be changed on line-breaks within not-black text (e.g. comments (green here))
  showspaces=false,                % show spaces everywhere adding particular underscores; it overrides 'showstringspaces'
  showstringspaces=false,          % underline spaces within strings only
  showtabs=false,                  % show tabs within strings adding particular underscores
  stepnumber=1,                    % the step between two line-numbers. If it's 1, each line will be numbered
  stringstyle=\color{mymauve},     % string literal style
  tabsize=4,                       % sets default tabsize to 2 spaces
  title=\lstname,                   % show the filename of files
}

\bibliography{references}
\begin{document}

\title{Numerical Simulation of an Expanding Universe \\ Numerical Methods Final Project}
\author{Carlos Rafael Salazar Letona \\ s21205751}
\maketitle

\section*{Introduction}
This simulation project is concerned with analyzing the structure formation in the universe. For this purpose, the \Lambda CDM model is utilized for creating a numerical approximation of our universe.

\subsection*{\Lambda CDM model}
The \Lambda CDM model is considered the standard model of cosmology, as it is the simplest theory that reproduces the most important results, such as \cite{ref:lambdacdm}:

\begin{enumerate}
\item the existence and structure of the CMB
\item the large-scale structure in the distribution of galaxies
\item the observed abundances of the lightest atoms
\item the accelerating expansion of the universe
\end{enumerate}
In this model, the energy of the universe is divided in three components: dark energy, cold dark matter, and ordinary matter.

To account for the scale of the universe, a dimensionless factor is used $a$. This factor is used together with the comoving position $\vec{x}$ to find the physical position $\vec{r}$, related by $\vec{r} = a \vec{x}$. Using the comoving position, two points will always have the same distance between them at any moment in time. The equations of motion can then be solved by taking derivatives of the physical position $\vec{r}$.

Then, using Poisson's equation for gravity, we can relate the gravitational potential to the matter distribution.

\begin{equation}
\Lambda_{r} \phi = 4 \pi G \rho_{m}
\end{equation}

By also considering General Relativity, one can extend the energy distribution to consider other components of the universe. For this project, dark matter is ignored, ordinary matter has no pressure, and dark energy has a negative pressure (accounting for the accelerating expansion rate of the universe).

\begin{equation}
\Lambda_{r} \phi = 4 \pi G \sum_{i} \rho_{i} (1 + 3w_{i}) = 4 \pi G \lcurl \rho_{m} - 2 \rho_{\Lambda} \rcurl
\end{equation}


\begin{figure}[H]
	\centering
	\includegraphics[width=0.8\linewidth]{normalDistributions.png}
	\caption{Different normal distributions and their corresponding mean and standard deviation squared (variance)\cite{ref:normalWiki}.}
	\label{ref:normalCurve}
\end{figure}

\subsection*{Power Spectrum}



\section*{Problem and modelling}
This project will study the structure formation in this universe. This will be done through an N-body simulation. An $N_{p}$ number of particles are placed inside a periodic box of size $L$. These particles can represent galaxies or other structures in the universe, and by numerically solving for their positions, it is possible to study how structures are formed.

In order to have a structure formation similar to the one in our universe, we need to have the right initial conditions. For this, we can use the power spectrum. The theoretical power spectrum can be used as a reference. If we set the particles such that they are equidistant from each other, we can then add small random displacements such that the power spectrum matches the theoretical power spectrum.


\section*{Approach and numerical methods used}
In order to check the Central Limit Theorem a Fortran code was written. The purpose of which was to get the distribution of the sum $S_{n}$ of $n$ random variables. In order to do this, a histogram representation was prepared using the code.

Realistically, the sum of the random variables cannot be calculated an infinite amount of times. For this reason, a numerical method is needed. For this purpose, the Monte Carlo method was used. The Monte Carlo methods rely on repeated random sampling to in order to get numerical results\cite{ref:monteCarloWiki}. In this case, the sum of the random variables is what is calculated many times, each time using new random numbers. Using this method, we will get an approximation to the distribution of $S_{n}$ that depends on the number of iterations.

The Fortran code proceded as follows:

\begin{enumerate}
	\item The variables are defined. In this case, $Sn$ will represent the sum of the random variables $x$. The variables $i$ and $b$ are used as counters for the for loops. The variable $n$ represents the number of random variables used for the sum, and $iterations$ is the number of times $Sn$ is calculated, i.e. the Monte Carlo part of the algorithm. Lastly, the variable $pos$ will point to the element of the array $bins$ that needs to be increased depending on the value of $Sn$. The size of $bins$ is set to be twice the value of $n$ in order to cover all possible values of $Sn$.
	\begin{lstlisting}
		real::Sn, x
		integer::i, b, n, iterations, pos
		!Define bins array size to be twice the value 
		!of n so that it covers all the possible values
		integer, dimension(100)::bins
	\end{lstlisting}
	\item The random number generator is then initiated with the default seed. The value of $n$ will be set to different values for different parts of the project, it will take the values of 5, 10, 50. The number of iterations of the Monte Carlo method is set to 10000000. The $bins$ array is also initialized to get a 0 count for each value. 
	\begin{lstlisting}
		!Initiate the random number generator
		call random_seed()

		n = 50
		iterations = 10000000

		!Give the bins an initial count of 0
		do i=0, 2*n
			bins(i) = 0
		enddo
	\end{lstlisting}
	\item Then, the actual Monte Carlo method can start. This consists in the first loop of the variable $i$ going from 0 to the number of iterations, resulting in many values of $Sn$. Each value of $Sn$ is then calculated using the second for loop of $b$ going from 0 to the number of variables $n$. Inside this loop a random number is obtained, which has a uniform distribution going from 0 to 1. This is then converted to a uniform distribution between -1 and +1, and the resulting number is added to the value of $Sn$. Once the $n$ variables have been added, the count in position i is increased, determined by $i < S_{n} < i+1$. 
	\begin{lstlisting}
		!Iterate to get different Sn values
		do i=0, iterations
			Sn = 0.0
			!Iterate to get the sum of x variables Sn
			do b=0, n
				call random_number(x)
				!Define the x variable to 
				!be between -1 and 1
				x = 2.0 * x - 1.0
				Sn = Sn + x
			enddo
			!Depending on the value of Sn 
			!put it in bin with position pos
			pos = int(floor(Sn + n + 1))
			bins(pos) = bins(pos) + 1
		enddo
	\end{lstlisting}
	\item The $bins$ array can then be stored in a file which is later used for plotting the data. 
	\begin{lstlisting}
		!Write the bin array into a file
		do i=1, 2*n
			write(2*n,*)bins(i)
		enddo
	\end{lstlisting}
\end{enumerate}


\section*{Analysis of the numerical results}

\subsection*{Growth of the linear part of the power spectrum}
To confirm that the power spectrum grows proportionally with the square of the growth factor, as implied by:

\begin{equation}
P(k) = \frac{ D(a)^{2}}{ D(a_{i})^{2}} P_{i}(k)
\end{equation}

This was done by performing a simulation of 50 realizations, with 11 saved snapshots, and an $\Omega_{m,0}$ value of 0.32. After averaging over the realizations, the power spectrum was plotted, as shown in fig~\ref{fig:spectrum05}. On the left, the calculated power spectrum from the simulation is seen. The initial and final power spectrum are shown in blue and red, and the snapshots in between are shown in green, with an increasing opacity over time. On the right, the initial theoretical spectrum can be seen, as well as the intermediate spectrums for the different values of the growth factor $D(a)$, and the final spectrum in red again. The plot of the initial and final spectrums from the simulation are overlayed to show indeed that they agree on the linear part, i.e., at the lower frequencies.

\begin{figure}[htp]
	\centering
	\begin{subfigure}[b]{0.9\textwidth}
		\centering
		\includegraphics[width=\textwidth]{Graphs/measured05.png}
		\caption{}
		\label{fig:measured05}
	\end{subfigure}
	\hfill
	\begin{subfigure}[b]{0.9\textwidth}
		\centering
		\includegraphics[width=\textwidth]{Graphs/theoretical05.png}
		\caption{}
		\label{fig:theoretical05}
	\end{subfigure}
	\caption{Measured and theoretical power spectrums}
	\label{fig:spectrum05}
\end{figure}




\section*{Conclusion}
In this project it was shown that the Central Limit Theorem can be checked using a numerical method, in this case, the Monte Carlo method. It was shown that the distribution of the sum of random variables $S_{n}$ tends towards a normal distribution with higher number of random variables. Then, it was proved that the average and standard deviation of the distribution of $S_{n}$ can be calculated using the average and standard deviation of the original distribution of the random variables, and that it will always tend to a normal distribution independently of the original distribution.


\section*{References}
\printbibliography[heading=none]

\section*{Appendix}
\subsection*{Multithreading}
As many types of N-body simulations, the code written for this project could highly benefit from properly utilizing a CPU with multiple cores. In order to implement multithreading, the "threading" class in C++ was used.
\begin{lstlisting}
	program centralLimit

	!Variables definition
	real::Sn, x
	integer::i, b, n, iterations, pos
	!Define bins array size to be twice the 
	!value of n so that it covers all the possible values
	integer, dimension(100)::bins

	!Initiate the random number generator
	call random_seed()

	n = 50
	iterations = 10000000

	!Give the bins an initial count of 0
	do i=0, 2*n
		bins(i) = 0
	enddo

	!Iterate to get different Sn values
	do i=0, iterations
		Sn = 0.0
		!Iterate to get the sum of x variables Sn
		do b=0, n
			call random_number(x)
			!Define the x variable to be between -1 and 1
			x = 2.0 * x - 1.0
			Sn = Sn + x
		enddo
		!Depending on the value of Sn 
		!put it in bin with position pos
		pos = int(floor(Sn + n + 1))
		bins(pos) = bins(pos) + 1
	enddo

	!Write the bin array into a file
	do i=1, 2*n
		write(2*n,*)bins(i)
	enddo

	end program
\end{lstlisting}

Code used for the uniform distribution between 0 and +1.
\begin{lstlisting}
	program centralLimit

	!Variables definition
	real::Sn, x
	integer::i, b, n, iterations, pos
	!Define bins array size to be equal to 
	!the value of n so that it covers all the possible values
	integer, dimension(25)::bins

	!Initiate the random number generator
	call random_seed()

	n = 25
	iterations = 10000000

	!Give the bins an initial count of 0
	do i=0, n
		bins(i) = 0
	enddo

	!Iterate to get different Sn values
	do i=0, iterations
		Sn = 0.0
		!Iterate to get the sum of x variables Sn
		do b=0, n
			!Obtain an x variable uniformly 
			!distributed between 0 and 1
			call random_number(x)
			Sn = Sn + x
		enddo
		!Depending on the value of Sn 
		!put it in bin with position pos
		pos = int(floor(Sn + 1))
		bins(pos) = bins(pos) + 1
	enddo

	!Write the bin array into a file
	do i=1, n
		write(n,*)bins(i)
	enddo

	end program
\end{lstlisting}

\end{document}



Without using threads
int Np = 2048;  //Number of particles
int timeSteps = 1000;  //Number of timesteps
int realisations = 50;  //Number of realisations to be performed
int snaps = 10;  //Number of snapshots to be saved
bool randPos = true;  //Bool to specify if positions should be randomized
double L = 1000;  //Periodic box size
double H0 = 100;  //H0 value
double omegam0 = 0.32;  //density
TOTAL EXECUTION TIME: 2734.69seconds
TIME PER REALISATION: 54.6938seconds


Using threads
int Np = 2048;  //Number of particles
int timeSteps = 1000;  //Number of timesteps
int realisations = 50;  //Number of realisations to be performed
int snaps = 10;  //Number of snapshots to be saved
bool randPos = true;  //Bool to specify if positions should be randomized
double L = 1000;  //Periodic box size
double H0 = 100;  //H0 value
double omegam0 = 0.32;  //density
TOTAL EXECUTION TIME: 1086 seconds
TIME PER REALISATION: 21.72 seconds

Around 2.5 times faster

Compilation  g++ -std=c++11 -pthread
g++ -std=c++11 -pthread simulation.cpp -o simulation